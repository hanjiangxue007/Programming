\documentclass{article}
\usepackage{scalerel}


\def\stretchint#1{\vcenter{\hbox{\stretchto[220]{\displaystyle\int}{#1}}}}
\def\scaleint#1{\vcenter{\hbox{\scaleto[3ex]{\displaystyle\int}{#1}}}}
\def\bs{\!\!}

%%%%%%%%%%%%%%%%%%%%%%%%%%%%%%%%%%%%%%%%%%%%%%%%%%%%%%%%%%%%%%%%%%%%%%%%%%%%%%%
\begin{document}
\section{Example 1}


\def\x{\frac{a}{c}dP}  % define the variable x = a/c dP

\verb|\stretchto| with aspect ratio limit of 2.2\par  % first sentence

\def\bs{\!\!\!\!}      % define variable bs = 4 spaces

\[
\int_a^b\x                          % first integral (~ is whitespace)
~~ {\stretchint{7ex}}_{\bs a}^b\x   % second integral
~~ {\stretchint{9ex}}_{\bs a}^b\x   % third integral
\]
\par{}


\verb|\scaleto| with width limit of 3ex\par  % third sentence
\def\bs{\!\!\!\!\!}

\[
\int_a^b\x ~~ {\scaleint{7ex}}_{\bs a}^b\x ~~ {\scaleint{9ex}}_{\bs a}^b\x
\]

% example 2
\section{Example 2}
$ \mu_{x_{t}} = E(x_{t}) = \int_{a}^{b} $                  % math mode
\[ \mu_{x_{t}} = E(x_{t}) = \int_{a}^{b} \]                % displaymath mode
$ \mu_{x_{t}} = E(x_{t}) = {\displaystyle \int_{a}^{b} } $ % both

%% example 3
\section{Example 3}
\begin{itemize}
    \item $ \mu_{x_{t}} = E(x_{t}) = \int_{a}^{b} $
    \item $ \mu_{x_{t}} = E(x_{t}) = \int\limits_{a}^{b} $ % same
    \item $ \mu_{x_{t}} = E(x_{t}) = \displaystyle\int_{a}^{b} $
    \item $ \mu_{x_{t}} = E(x_{t}) = \displaystyle\int\limits_{a}^{b} $ % same
\end{itemize}

\end{document}
