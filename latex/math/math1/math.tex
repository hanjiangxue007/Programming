%Author: Bhishan Poudel
% Date : Aug, 2015
% cmd  : latex poudel1.tex 
% cmd  : xdvi poudel1.dvi 
% cmd  : dvipdf poudel1.dvi


\documentclass[12pt]{article}
\usepackage{amsmath} % for equation*

%some useful newcommands
\newcommand{\beq}{\begin{equation}}
\newcommand{\eeq}{\end{equation}}
\newcommand{\beqa}{\begin{eqnarray}}
\newcommand{\eeqa}{\end{eqnarray}}
\newcommand{\beqan}{\begin{eqnarray*}}
\newcommand{\eeqan}{\end{eqnarray*}}

% some useful newcommands
\newcommand{\nl}{\nonumber \\}
\newcommand{\no}{\nonumber}
\newcommand{\ul}{\underline}
\newcommand{\ol}{\overline}

\begin{document}

\section{example}
	\beqa
	\label{eq1}
	z &=& x + iy\\
	\label{eq2}
	z &=& re^{i \phi}  \no\\   % do not number this equation
	z &=& r cos(\phi) + r sin(\phi)\\
	\sqrt{z} &=& \sqrt{r} cos(\phi/2) + \sqrt{r}  sin(\phi/2)
	\eeqa


\section{ example}

	Thus for all real numbers $x$ we have
	\begin{equation*}
    x\le|x|\quad\text{and}\quad x\ge|x|
	\end{equation*}
	
	and so
	\begin{equation*}
    x\le|x|\quad\text{for all $x$ in $R$}.
	\end{equation*}



\end{document}
