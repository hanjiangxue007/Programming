%cmd: clear; latex mymath.tex
%cmd: clear; xdvi mymath.dvi



\documentclass{article}
\usepackage{amsmath}

\title{Mathematics}
\author{Bhishan Poudel}
\date{August 2015}

\begin{document}
\maketitle

\section{ Section   1 Tex style }
The equation representing
a straight line in the Cartesian plane
is of the form $ax+by+c=0$, where $a$, $b$, $c$ are constants.

\section{ Section   2 Latex style}
The equation representing a straight line in the Cartesian plane is of
the form \(ax+by+c=0\), where \(a\), \(b\), \(c\) are constants.

\section{ Section   3 Latex style}
The equation representing a straight line in the Cartesian plane is
of the form \begin{math}ax+by+c=0\end{math}, where \begin{math} a
\end{math}, \begin{math} b \end{math}, \begin{math} c \end{math} are
constants.

\section{ Section   4}
The equation representing a straight line in the Cartesian plane is
of the form
$$
ax+by+c=0
$$
where $a$, $b$, $c$ are constants.

\section{ Section   5}
In the seventeenth century, Fermat conjectured that if $n>2$, then
there are no integers $x$, $y$, $z$ for which
$$
x^n+y^n=z^n.
$$
This was proved in 1994 by Andrew Wiles.

\section{ Section   6}

Numbers of the form $2^{2^n}+1$, where $n$ is a natural number, are
called Fermat numbers.

\section{ Section   7}

The sequence $(x_n)$ defined by
$$
x_1=1,\quad x_2=1,\quad x_n=x_{n-1}+x_{n-2}\;\;(n>2)
$$
is called the Fibonacci sequence.

\section{ Section   8}
If the sequence $(x_n)$ converges to $a$, then the sequence
$(x_n^2)$ converges to $a^2$

\section{ Section   9}
$$
x_m^n\qquad x^n_m\qquad {x_m}^n\qquad {x^n}_m
$$

\section{ Section   10}
The sequence
$$
2\sqrt{2}\,,
\quad 2^2\sqrt{2-\sqrt{2}}\,,
\quad 2^3\sqrt{2-\sqrt{2+\sqrt{2}}}\,,
\quad 2^4\sqrt{2-\sqrt{2+\sqrt{2+\sqrt{2+\sqrt{2}}}}}\,,
\;\ldots
$$
converge to $\pi$.
%%% The \ldots command above produces . . .

\section{ Section   11}
\newcommand{\vect}[1]{(#1_1,#1_2,\dots,#1_n)}
$\vect{x}$

\section{ Section   12}
Thus for all real numbers $x$ we have
\begin{equation*}
    x\le|x|\quad\text{and}\quad x\ge|x|
\end{equation*}
and so
\begin{equation*}
    x\le|x|\quad\text{for all $x$ in $R$}.
\end{equation*}

\section{ Section   13}
\begin{equation*}
\begin{split}
(a+b+c+d+e+f)ˆ2 & = aˆ2+bˆ2+cˆ2+dˆ2+eˆ2+fˆ2\\
    &\quad +2ab+2ac+2ad+2ae+2af\\
    &\quad +2bc+2bd+2be+2bf\\
    &\quad +2cd+2ce+2cf\\
    &\quad +2de+2df\\
    &\quad +2ef
\end{split}
\end{equation*}

\section{ Section   14}
Thus $x$, $y$ and $z$ satisfy the equations
\begin{align*}
x+y-z & = 1\\
x-y+z & = 1
\end{align*}

\section{ Section   15 compare equations}
\begin{align*}
\cos^2x+\sin^2x & = 1
&\qquad \cosh^2x-\sinh^2x & = 1\\
\cos^2x-\sin^2x & = \cos 2x &\qquad \cosh^2x+\sinh^2x & = \cosh 2x
\end{align*}

\section{ Section   16 (cases) multiples cases}
\begin{equation*}
|x| =
\begin{cases}
x & \text{if $x\ge 0$}\\
-x & \text{if $x\le 0$}
\end{cases}
\end{equation*}




\end{document}
