% topic: latex template for homework

% cmd  : clear; latex chick.tex  (Note: it will create 3 extra files log,aux and dvi )
% cmd  : clear; xdvi chick.dvi
% cmd  : clear; dvipdf chick.dvi

% cmd  : ctrl D if problem occurs
% NOTE : look at end for description of figure in LATEX.


\documentclass[14pt]{article}
\textheight 655pt
\textwidth 16.5cm
\hoffset -1.8cm  \voffset -1.7cm 

\usepackage{graphicx,subfigure}
\usepackage{textcomp}
\usepackage{amssymb}
\usepackage{fixltx2e}
\usepackage{xcolor}

\newcommand{\beq}{\begin{equation}}
\newcommand{\eeq}{\end{equation}}
\newcommand{\bfig}{\begin{figure}}
\newcommand{\efig}{\end{figure}}
\newcommand{\beqa}{\begin{eqnarray}}
\newcommand{\eeqa}{\end{eqnarray}}
\newcommand{\beqan}{\begin{eqnarray*}}
\newcommand{\eeqan}{\end{eqnarray*}}
\newcommand{\ba}{\begin{array}}
\newcommand{\ea}{\end{array}}
\newcommand{\ben}{\begin{enumerate}}
\newcommand{\een}{\end{enumerate}}
\newcommand{\bfl}{\begin{flushleft}}
\newcommand{\efl}{\end{flushleft}}
\newcommand{\btab}{\begin{tabular}}
\newcommand{\etab}{\end{tabular}}
\newcommand{\bit}{\begin{itemize}}
\newcommand{\eit}{\end{itemize}}
\newcommand{\bdes}{\begin{description}}
\newcommand{\edes}{\end{description}}
\newcommand{\bdm}{\begin{displaymath}}
\newcommand{\edm}{\end{displaymath}}
\newcommand {\IR} [1]{\textcolor{red}{#1}}


% Creating Title for the assesment

\title{Assignment 2}
\author{Bhishan Poudel}
\date{Sep, 2015}
% Don't forget to use \maketitle below \begin{document}

\begin{document}
\maketitle



\section{Section 1}
This is a sample text.
\clearpage



\begin{figure}[ht!]
\centering
\includegraphics[scale=0.7]{chick}
\caption{fig.1}
\end{figure}



\section{Section 2}

\setlength{\unitlength}{0.8cm}
\begin{picture}(6,5)
\put(3.5,0.4){$\displaystyle
s:=\frac{a+b+c}{2}$}
\put(1,1){\includegraphics[
  width=2cm,height=2cm]{chick.eps} }
\end{picture}


\section{Section 3}
\begin{figure}[ht!]
\centering
\includegraphics[scale=0.5, angle=180]{chick}
\caption{fig.1}
\end{figure}


\end{document}

%%%%%%%%%%%%%%%%%%%%%%%%%%%%%%%%%%%%%%%%%%%%%%%%%%%%%%%%%%%%%%
%\includegraphics[attr1=val1, attr2=val2, ..., attrn=valn]{imagename}
%\DeclareGraphicsExtensions{.pdf,.png,.jpg}

%	width=xx 			Specify the preferred width of the imported image to xx.
%						NB. Only specifying either width or height will scale the 
%						image while maintaining the aspect ratio.
%	height=xx 			Specify the preferred height of the imported image to xx.
%	keepaspectratio 	This can be set to either true or false. When true, it 
%						will scale the image according to both height and width, 
%						but will not distort the image, so that neither width nor height are exceeded.
%	scale=xx 			Scales the image by the desired scale factor. 
%						e.g, 0.5 to reduce by half, or 2 to double.
%	angle=xx 			This option can rotate the image by xx degrees (counter-clockwise)
%	trim=l b r t 		This option will crop the imported image by l from the left, 
%						b from the bottom, r from the right, and t from the top. 
%						Where l, b, r and t are lengths.
%	clip 				For the trim option to work, you must set clip=true.
%	page=x 				If the image file is a pdf file with multiple pages, 
%						this parameter allows you to use a different page than the first.
%	resolution=x 		Specify image resolution in dpi

%	\includegraphics[width=\linewidth]{chick}
%	\includegraphics[width=\textwidth]{chick}
%	\includegraphics[height=\textheight]{chick}
%	\includegraphics[scale=0.5, angle=180]{chick}
%	\includegraphics[trim = 10mm 80mm 20mm 5mm, clip, width=3cm]{chick}

%	Float placement specifiers
%	=============================

%    ! indicates that some restrictions should be ignored
%    h indicates that the float is allowed to be placed inline
%    t indicates that the float is allowed to go into a top area
%    b indicates that the float is allowed to go into a bottom area
%    p indicates the the float is allowed to go on a float page or column area

%	
%    If the specifier contains a !, the algorithm will ignore any restrictions 
%    related either to the number of floats that can be put into an area or the 
%    max size an area can occupy. Otherwise the restrictions defined by the parameters apply.
%    
%    As a next step it will check if h has been specified.
%    If so, it will try to place the float right where it was encountered. 
%    If this works, i.e., if there is enough space, 
%    then it will be placed and processing of that floats ends.
%    
%    If not, it will look next for t and if that has been specified it will try 
%    to place the float in the top area. If there is no restriction that prevents 
%    it then the float is placed and processing stops.
%    
%    If not it will finally check if b is present and, if so, it will try to place 
%    the float into the bottom area 
%    (again obeying any restrictions that apply if ! wasn't given).
%    If that doesn't work either or is not permitted because the specifier wasn't 
%    given the float is added to the holding queue.
%    
%    A p specifier (if present) is not used during the above process. 
%    It will only be looked at when the holding queue is being emptied at the next page boundary.







