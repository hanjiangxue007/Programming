\documentclass[12pt]{article}
\usepackage[english]{babel}
\usepackage[utf8x]{inputenc}
\usepackage{fullpage}

\usepackage{listings}
\usepackage{color}

\title{Syntax Highlighting in LaTeX with the listings Package}
\author{writeLaTeX}

\definecolor{mygreen}{rgb}{0,0.6,0}
\definecolor{mygray}{rgb}{0.5,0.5,0.5}
\definecolor{mymauve}{rgb}{0.58,0,0.82}

\lstset{ %
  backgroundcolor=\color{white},   % choose the background color
  basicstyle=\footnotesize,        % size of fonts used for the code
  breaklines=true,                 % automatic line breaking only at whitespace
  captionpos=b,                    % sets the caption-position to bottom
  commentstyle=\color{mygreen},    % comment style
  escapeinside={\%*}{*)},          % if you want to add LaTeX within your code
  keywordstyle=\color{blue},       % keyword style
  stringstyle=\color{mymauve},     % string literal style
}

\begin{document}

\maketitle

\section{Java}
\begin{lstlisting}[language=java]
class HelloWorldApp {
    public static void main(String[] args) {
        System.out.println("Hello World!"); // Display the string.
        for (int i = 0; i < 100; ++i) {
            System.out.println(i);
        }
    }
}
\end{lstlisting}

\section{Python}
% from http://wiki.scipy.org/Numpy_Example_List
\begin{lstlisting}[language=python]
>>> from numpy import *
>>> from numpy.fft import *
>>> signal = array([-2., 8., -6., 4., 1., 0., 3., 5.])
>>> fourier = fft(signal)
>>> N = len(signal)
>>> timestep = 0.1 # if unit=day -> freq unit=cycles/day
>>> freq = fftfreq(N, d=timestep) # freqs corresponding to 'fourier'
>>> freq
array([ 0. , 1.25, 2.5 , 3.75, -5. , -3.75, -2.5 , -1.25])
>>> fftshift(freq) # freqs in ascending order
array([-5. , -3.75, -2.5 , -1.25, 0. , 1.25, 2.5 , 3.75])
\end{lstlisting}

\section{MATLAB/Octave}
% from http://wiki.scipy.org/Numpy_Example_List
\begin{lstlisting}[language=octave]
octave:1> function xdot = f (x, t)
>
>  r = 0.25; k = 1.4;
>  a = 1.5; b = 0.16; c = 0.9; d = 0.8;
>
>  xdot(1) = r*x(1)*(1 - x(1)/k) - a*x(1)*x(2)/(1 + b*x(1));
>  xdot(2) = c*a*x(1)*x(2)/(1 + b*x(1)) - d*x(2);
>
> endfunction
\end{lstlisting}

\end{document}