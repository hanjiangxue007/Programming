
%cmd: clear; latex indent.tex
%cmd: clear; xdvi indent.dvi

% colors : white, black, red, green, blue, cyan, magenta, yellow.
% dvips color names:


\documentclass[12pt]{article}
\usepackage[utf8]{inputenc}
\usepackage[english]{babel}
\usepackage{graphicx}

\usepackage{verbatim} % for formatting verbatim
\usepackage{alltt}  % for formatting
\usepackage[usenames,dvipsnames,svgnames,table]{xcolor}


\newcommand{\comment}[1]{}  % for commenting \comment{This is a long comment }

%\usepackage{color}

\title{Assignment 1}
\author{Bhishan Poudel}
\date{August 2015}

\begin{document}


\setlength\parindent{24pt}

I am writing my Thesis report using LaTeX, I need to add indentation because 
every thing is right but only thing is every new paragraph is starting at 
initial stage but I need to add some indentation. 


\begin{itemize}
    \item This is the first line \\ %line with dot
          This is the second line   %line without dot 
    \item Next line with dot
\end{itemize}




\begin{center}
Example 1: The following paragraph (given in quotes) is an 
example of Center Alignment using the center environment. 
 
``LaTeX is a document preparation system and document markup 
language. LaTeX uses the TeX typesetting program for formatting 
its output, and is itself written in the TeX macro language. 
LaTeX is not the name of a particular editing program, but 
refers to the encoding or tagging conventions that are used 
in LaTeX documents".
\end{center}

This is the text in first paragraph. This is the text in first 
paragraph. This is the text in first paragraph. \par
This is the text in second paragraph. This is the text in second 
paragraph. This is the text in second paragraph.




\begin{flushleft}
``LaTeX is a document preparation system and document markup 
language. LaTeX uses the TeX typesetting program for formatting 
its output, and is itself written in the TeX macro language. 
LaTeX is not the name of a particular editing program, but refers 
to the encoding or tagging conventions that are used in LaTeX documents".
\end{flushleft}

\setlength{\parindent}{10ex}
This is the text in first paragraph. This is the text in first 
paragraph. This is the text in first paragraph. \par
\noindent %The next paragraph is not indented
This is the text in second paragraph. This is the text in second 
paragraph. This is the text in second paragraph.




\begin{verbatim}
The verbatim environment
  simply reproduces every
 character you input,
including all  s p a c e s!
Verbatim doesnot allow other commands
but alltt allows
\end{verbatim}


\begin{alltt}
Verbatim extended with the ability
to use normal commands.  Therefore, it
is possible to \emph{emphasize} words in
this environment, for example.
\end{alltt}


This is another
\begin{comment}
rather stupid,
but helpful
\end{comment}
example for embedding
comments in your document.














\end{document}


%	Remember to add \usepackage{alltt} to your preamble to use it though! 
%	Within the alltt environment, you can use the command \normalfont to 
%	get back the normal font. To write equations within the alltt enviroment, 
%	you can use \( and \) to enclose them, instead of the usual $.

%	When using \textbf{} inside the alltt enviroment, note that the standard 
%	font has no bold TT font. Txtfonts has bold fonts: just add 
%	\renewcommand{\ttdefault}{txtt} after \usepackage{alltt}.

%	If you just want to introduce a short verbatim phrase, you don't need to 
%	use the whole environment, but you have the \verb command:


%	Paragraph indention is controled by the parameter \parindent. 
%	In most document classes it is set to a positive value so you 
%	should see indentations. If this is not the case you can set 
%	this parameter in the document preamble to whatever value you wish, 
%	e.g.

%	\setlength\parindent{24pt}

%	Of course, a requirement is that you mark up your paragraphs: 
%	a paragraph ends by either a blank line or by the command \par. 
%	If you instead just used \\you have directed LaTeX to start a new 
%	line but not a new paragraph.

