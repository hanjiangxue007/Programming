% Author : Bhishan Poudel
% Date   : Apr 11, 2016
% Ref    : https://tex.stackexchange.com/questions/172736/place-parentheses-around-a-cross-reference
\documentclass{article}
\usepackage{amsmath} % for eqref
% reference within parenthesis
\let\oldref\ref
\renewcommand{\ref}[1]{(\oldref{#1})}

% for customized autoref for hyperref
\usepackage{hyperref}
\def\equationautorefname~#1\null{Eq. (#1)\null}

% Creating Title
\title{Cross Reference in Latex}
\author{Bhishan Poudel}
\date{Apr 11, 2016}

%%***************************** document begins *****************************%%
\begin{document}
\maketitle
\tableofcontents
\listoffigures
\clearpage

\section{Reference inside parenthesis}\label{sec:testlabel}
We always compile twice to see the reference number.\\
Always put label under the caption of the figure.\\
Always give the prefix to the labels.\\
Referencing tags: \\
ch: 	chapter \\
sec: 	section\\
subsec: 	subsection\\
fig: 	figure\\
tab: 	table\\
eq: 	equation\\
lst: 	code listing\\
itm: 	enumerated list item\\
alg: 	algorithm\\
app: 	appendix subsection\\

Method 1: using newcommand\\
Reference: \ref{sec:testlabel} \\

Method 2: using amsmath eqref\\
Reference: \eqref{sec:testlabel} \\

\section{Formulae}
\label{sec:Formulae}

Here is an example showing how to reference formulae:\\

\begin{equation} \label{eq:solve}
x^2 - 5 x + 6 = 0
\end{equation}

\begin{equation}
x_1 = \frac{5 + \sqrt{25 - 4 \times 6}}{2} = 3
\end{equation}

\begin{equation}
x_2 = \frac{5 - \sqrt{25 - 4 \times 6}}{2} = 2
\end{equation}

and so we have solved equation~\autoref{eq:solve}.


\end{document}