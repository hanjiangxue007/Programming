\documentclass{article}
\usepackage{listings}
\usepackage[usenames,dvipsnames]{color}  %% Allow color names
\lstdefinestyle{customasm}{
  belowcaptionskip=1\baselineskip,
  xleftmargin=\parindent,
  language=C++,   %% Change this to whatever you write in
  breaklines=true, %% Wrap long lines
  basicstyle=\footnotesize\ttfamily,
  commentstyle=\itshape\color{Gray},
  stringstyle=\color{Black},
  keywordstyle=\bfseries\color{OliveGreen},
  identifierstyle=\color{blue},
  xleftmargin=-8em,
}
\usepackage[colorlinks=true,linkcolor=blue]{hyperref}
\begin{document}
\tableofcontents

\newpage
\section{a.tex}
\lstinputlisting[style=customasm]{a.tex}
\newpage
\section{hello.py}
\lstinputlisting[style=customasm]{hello.py}
\newpage
\section{remove_whitespaces.py}
\lstinputlisting[style=customasm]{remove_whitespaces.py}
\newpage
\section{res.ps}
\lstinputlisting[style=customasm]{res.ps}
\newpage
\section{result.txt}
\lstinputlisting[style=customasm]{result.txt}
\newpage
\section{result2.txt}
\lstinputlisting[style=customasm]{result2.txt}
\newpage
\section{src2psf.sh}
\lstinputlisting[style=customasm]{src2psf.sh}
\end{document}
